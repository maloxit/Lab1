\subsection{Поиск $\delta$}

Рассмотренная нами матрица является неособенной при малых значения $\delta$. Матрица становятся особенными, когда интервалы, находящиеся в столбцах, имеют не пустое пересечение. Более того, если происходит пересечение интервалов во всех столбцах определенных строк, то тогда можно выделить такую точечную матрицу, из этих интервалов, в которой определитель будет равен 0. Иначе говоря, в таком случае определитель матрицы интервалов содержит в себе 0.

\subsection{Поиск глобального минимума}

Для обеих функций алгоритм нахождения минимума функции дал правильную оценку значения минимума функции, при этом чуть хуже оценил аргументы, сообщающие минимум. При этом видно, что для функции, имеющей несколько равнозначных глобальных минимумов наблюдаются скачки между этими самыми локальными минимумами. Скачкообразное поведение графиков объясняется самим алгоритмом: ведущий брус, который будет далее дробиться, выбирается на каждой итерации путем полного итерирования по рабочему списку, то есть не обязательно последний брус дает наилучшее приближение к минимуму.