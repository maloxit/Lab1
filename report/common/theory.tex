\subsection{Критерий Баумана}
Матрица \textbf{A} $\in {\mathbb{R}}^{n \times n}$ неособенна \Leftrightarrow \forall A', A'' $\in$ vert \textbf{A} det(A') · det(A'') > 0 

\subsection{Глобальная оптимизация}
Суть простейшего интервального адаптивного алгоритма глобальной оптимизации похожа на алгоритм дихотомии, только для многомерного случая. Имеется рабочий список рассматриваемых брусьев, для каждого из которых вычислено целевое значение функции (в интервальном смысле). На каждой итерации метод выбирает из этого списка брус, на котором нижняя оценка значения функции наименьшая. Этот брус удаляется из списка, после чего туда добавляются два новых, которые получились из исходного путем дробления его самой длинной компоненты пополам (от нижней границы до середины и от середины до верхней границы). На этих брусьях вычисляется интервальная оценка целевой функции, выполняется переход на новую итерацию.
