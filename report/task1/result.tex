\subsection{Задача 1: поиск $\delta$}

\subsubsection{Задача 1.1: 0 $\in$ det(\textbf{A})}

Определитель матрицы \textbf{A} 2$\times$2 вычисляется по следующей формуле:

\begin{equation}
	det(\textbf{A}) = a_{11}a_{22} - a_{12}_{21}
\end{equation}

Несложно убедиться, что для поставленной матрицы А (\ref{eq1}) определитель следующий:

\begin{equation}
	det(\textbf{A}) = [0.1 - 2\delta; 0.1 + 2\delta]
\end{equation}

Так как $\delta$ должен быть не меньше 0, то тогда det(\textbf{A}) будет содержать 0 тогда и только тогда, когда нижняя граница интервала будет меньше или равна 0; это достигается при $\delta \geq$ 0.05.

Таким образом, $\forall \delta \geq 0.05: $ 0 $\in$ det(\textbf{A}).

\subsubsection{Задача 1.2: матрица \textbf{A} - особенная}

Для определения "особенности" матрицы А (\ref{eq1}) при определенном $\delta$ воспользуемся критерием Баумана. Множество vert(\textbf{A}) состоит из 4 элементов:
\begin{equation}
	vert(\textbf{A}) = 
	                \begin{Bmatrix}
	                    \begin{pmatrix}
                            1.05 \pm $\delta$ & 1\\
                            0.95 \pm $\delta$ & 1
                        \end{pmatrix}
                    \end{Bmatrix}
\end{equation}

В силу малой мощности множества vert(\textbf{A}), допустимо посчитать все возможные det(A) A $\in$ vert(\textbf{A}):
\begin{enumerate}
    \item
	    det\begin{pmatrix}
                1.05 - $\delta$ & 1\\
                0.95 - $\delta$ & 1
            \end{pmatrix}
         = 0.1
         
    \item
	    det\begin{pmatrix}
                1.05 - $\delta$ & 1\\
                0.95 + $\delta$ & 1
            \end{pmatrix}
         = 0.1 - 2$\delta$
         
    \item
	    det\begin{pmatrix}
                1.05 + $\delta$ & 1\\
                0.95 - $\delta$ & 1
            \end{pmatrix}
         = 0.1 + 2$\delta$
        
    \item
	    det\begin{pmatrix}
                1.05 + $\delta$ & 1\\
                0.95 + $\delta$ & 1
            \end{pmatrix}
         = 0.1
\end{enumerate}

    Видно, что все определители матриц из vert(\textbf{A}) будут положительными только в случае, когда $\delta < 0.05$ и $\delta \geq 0$.
    Таким образом, $\forall \delta < 0.05$ и  $\delta \geq 0: $\textbf{A} - неособенная.